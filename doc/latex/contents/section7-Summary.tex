\section{Summary and Outlook} \label{sec:7-summary}
\subsection{Summary}
In summary, during this project we studied the problem of modelling and simulating and evacuation procedure scenario through the social force model. We studied relevant literature and then stated our specific research questions regarding the key parameters that we considered most important. We defined formally the model that we used and then implemented it. We conducted several experiments to test the behaviour of said model and to answer the research questions that we had stated.

The answers to our four research questions are summarized below:
\begin{enumerate}
	\item A higher \emph{desired velocity} caused the agents to try and move with a higher velocity.
	
	\item A higher \emph{desired velocity} leads to accelerated evacuation. However, starting from certain values of the desired velocity ($\SI{1.67}{\meter\second^{-1}}$ in our simulation), the influence of this parameter decreases.
	
	\item A lower \emph{panic level} caused the agents to behave in a more individualistic way, whereas a higher \emph{panic level} caused the agents to behave in a more herd-like way.
	
	\item There was a positive correlation between the \emph{panic level} and the evacuation time.
	
	\item A lower \emph{excitement factor} caused the agents to make well-informed decisions, whereas a higher \emph{excitement factor} caused the agents to make hastier and ill-informed decisions.
\end{enumerate}

\subsection{Further Ideas}
Some further ideas that could be explored are:
\begin{itemize}
	\item An injury mechanism if the total force on an agent is too large. Furthermore, the injured agent could work as an obstacle to other agents or have different behaviour and parameters.
	
	\item A memory mechanism for the agents, so that they consider previously seen exits instead of just their current field of view.
	
	\item A different pathfinding algorithm that considers the density of agents at every point and not just around exit points.
\end{itemize}