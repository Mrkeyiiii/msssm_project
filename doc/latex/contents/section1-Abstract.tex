
\documentclass{article}
\begin{document}
\section{Abstract}


Metro-station during the rush hours, shopping mall on weekends, ETH Mensa at lunch time.... With the development of society, similar situations that could be represented by scenarios mentioned  above are becoming a common part of our daily life. Therefore, precise predictions of the movement of the crowd through computational modelling, especially in threatening situations,i.e during evacuation, are of increasing importance, which eventually motivated us to lean our work on the topic evacuation modelling.\\
As a matter of fact, numerous researchers have already focused on the modelling of evacuation process with the well-accepted social force model. However, they all show their individual limitations and could be further improved regarding calculation time, calibration of the parameters, etc. \\
Therefore, we decide to base our work on a framework of a previous MSSSM group, who mainly based their work on [2]. Combined with [3], we will firstly try to find how various sets of parameters affact the behavior of the model and then see how we could improve the model by following the information mentioned in the previous studies, especially in [3].

\end{document}