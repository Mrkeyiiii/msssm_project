\section{Implementation}
The implementation of the model presented in Section \ref{sec:4-model} was based on the code from~\citet{Hardmeier2012}. Since the code of~\citet{Hardmeier2012} implements the social force model exactly as it is presented in~\citet{Helbing2000}, it was appropriate to take the code of~\citet{Hardmeier2012} as a starting point and make the necessary additions and alterations according to the model modifications of~\citet{Zainuddin2010} and~\citet{Wang2016}. Most of these additions and alterations warrant no particular explanation, and in general were made according to the model of Section~\ref{sec:4-model}.

The point that will be discussed in detail is the calculation of the congested radii $r_k(t)$ around each exit, from \eqref{eq:Uik}. This point was not discussed at all in~\citet{Zainuddin2010} or~\citet{Wang2016} and therefore a new implementation was needed.

Our congestion radius $r_k$ is defined as the radius of the largest semi-circle around the exit such that the density of people inside is larger than some user defined congestion density $\rho_{\textrm{congestion}}$. This can be calculated efficiently using algorithm~\ref{alg:r_congestion}.

\begin{algorithm}
	\label{alg:r_congestion}
	\caption{Computing the congestion radius for exit $k$}
	\begin{algorithmic}
		\State $\delta_i \gets sorted((d_{ik})_{i=1}^{N})$   \Comment{$d_{ik}$: distance of actor $\#i$ to the center of exit $\#k$}
		\State $r_k \gets 0$
		\For{$i \gets 1,\dots,N$}
			\If{$2 \dot i / \delta_i^2 \geq \rho_{\textrm{congestion}}$}
				\State $r_k \gets \delta_i$
			\Else
				\State \Return{$r_k$}
			\EndIf
		\EndFor{}
		\State \Return{$r_N$}
	\end{algorithmic}
\end{algorithm}

The movement was also slightly changed from the original model to include path finding. This allows for non-convex rooms. As outlined in Section~\ref{sec:4-model}, the direction of each actor now depends on some distance field $d_k$. This distance field is similar to the one described in \citet{Hardmeier2012}, but now each exit has it's own distance field because actors can choose exits by themselves. The distance field is computed explicitly along a grid, and interpolated linearly in-between. Computing this distance field can be done efficiently using \emph{Fast Sweeping}, reusing most of the code by \citeauthor{Hardmeier2012}. 

