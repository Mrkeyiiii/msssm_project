\section{Simulation Results and Discussion}
All simulations were carried out using the forward Euler method to numerically solve the differential equations of our model. The step size of the solver was constant for each simulation and was chosen so that the corresponding simulation was stable. The parameters that were used for the simulations are presented in Table \ref{tab:param}, most of which were adapted from~\citet{Wang2016} (with few alterations, wherever needed). The values of said parameters stem both from empirical data and from expected values of the quantities they represent, for further details regarding this point the reader is referred to~\citet{Helbing2000},~\citet{Zainuddin2010} and~\citet{Wang2016}. Any parameter that is not present in Table \ref{tab:param} or that was changed for a particular simulation will be mentioned in the corresponding simulation subsection.

\begin{table}[ht!]
	\, \hfill
	\begin{tabular}{c c c}
		\hline
		Parameter & Value & Unit \\ \hline
		$r_i$ & $[0.25 \,, 0.35]$ & \si{\meter} \\
		$r_i^p$ & 2 & \si{\meter} \\
		$m_i$ & $[50 \,, 70]$ & \si{\kilogram} \\
		$\tau$ & $0.5$ & \si{\second} \\
		$\nu_i^0$ & $1.5$ & \si{\meter\,\second^{-1}} \\
		$A_i$ & $200$ & \si{\newton} \\
		$C_i$ & $-200$ & \si{\newton} \\
		$k$ & $\num{1.4e4}$ & \si{\kilogram\,\second^{-2}} \\
		$\kappa$ & $\num{1e4}$ & \si{\kilogram\,\meter^{-1}\,\second^{-1}} \\
	\end{tabular}
	\hfill
	\begin{tabular}{c c c}
		\hline
		Parameter & Value & \\ \hline
		$N$ & 400 & \\
		$B_i$ & $0.1$ &  \\
		$D_i$ & $0.5$ &  \\
		$l_i$ & $0.05$ &  \\
		$\alpha_i$ & $0.3$ &  \\
		$\beta_i$ & $1.5$ &  \\
		$g_i$ & $0.2$ &  \\
		$E_i$ & $0.6$ &  \\
		$p_i$ & $0.3$ &  \\
	\end{tabular}
	\hfill \,
	\caption{Simulation parameters' values. A range of values instead of a singular value denotes that the values of that particular parameter were chosen randomly for each agent using a uniform distribution on that range. A singular value for a parameter that is agent-dependent denotes that all agents have the same parameter value.}
	\label{tab:param}
\end{table}


\subsection{Desired Velocity}


\subsection{Panic Level}
In order to determine how the panic level of the agents' affect their behaviour and the evacuation time, multiple simulations were conducted with various panic levels $p_i$. The panic level that was used was the same for all agents in each simulation, and therefore will be henceforth denoted as just $p$. The simulations were conducted in an arena with two exits of width $\SI{4.7}{\meter}$, $N=600$ agents, and the parameter $\alpha_i=0.6$ for all agents was used.

The panic level $p \in [0,1]$ as it appears in \eqref{eq:vi} affects how much each agent's speed is affected by the speed of the agents' speed around him (in a $\SI{2}{\meter}$ radius), therefore it is a measure of how susceptible each agent is to the behaviour of everyone around him. A value of $p=0$ would mean that each agent behaves in a completely individualistic way, whereas a value of $p=1$ would mean that the agents behave in a pure herding way. In general, in an urgent evacuation scenario where the agents are considerably panicked the panic level $p$ will be high, whereas in a relaxed non-urgent evacuation scenario the panic level will be low.

First, an initial simulation for the singular value of $p = 0.4$ was conducted. Six frames from the $\SI{60}{\second}$ simulation are presented in Figure \ref{fig:demo_P-0.4}. At $t=\SI{0}{\second}$, Figure \ref{fig:demo_P-0.4_a}, each agent evaluates the best exit and starts moving towards it. At $t=\SI{4}{\second}$, Figure \ref{fig:demo_P-0.4_b}, the agents have separated into two distinct groups, one heading towards the green exit and one heading towards the teal exit. At $t=\SI{8}{\second}$, Figure \ref{fig:demo_P-0.4_c}, the agents have reach both exits and they start evacuating. At $t=\SI{12}{\second}$, Figure \ref{fig:demo_P-0.4_d}, it can be seen that an arch-like blocking of both exits has taken place. At $t=\SI{16}{\second}$, Figure \ref{fig:demo_P-0.4_e}, some of the agents in the congestion around the exit have evacuated, although at a slow pace because of the arch-like congestion around the exits. At $t=\SI{20}{\second}$, Figure \ref{fig:demo_P-0.4_f}, almost all the agents have evacuated. The whole evacuation procedure, until all $600$ agents have evacuated the arena, takes $\SI{21.44}{\second}$. The arch-like blocking of the exits is consistent with empirical observations of evacuation procedures of large groups of people~\cite{Helbing2000}.

\begin{figure}[ht!]
	\begin{subfigure}{.48\textwidth}
		\centering
		\includegraphics[width=.9\linewidth]{{demo_P-0.4_0000_0.000}.eps}
		\caption{$t = \SI{0}{\second}$}
		\label{fig:demo_P-0.4_a}
	\end{subfigure}
	\begin{subfigure}{.5\textwidth}
		\centering
		\includegraphics[width=.9\linewidth]{{demo_P-0.4_0200_4.000}.eps}
		\caption{$t = \SI{4}{\second}$}
		\label{fig:demo_P-0.4_b}
	\end{subfigure}

	\vspace{\baselineskip}
	
	\begin{subfigure}{.48\textwidth}
		\centering
		\includegraphics[width=.9\linewidth]{{demo_P-0.4_0400_8.000}.eps}
		\caption{$t = \SI{8}{\second}$}
		\label{fig:demo_P-0.4_c}
	\end{subfigure}
	\begin{subfigure}{.5\textwidth}
		\centering
		\includegraphics[width=.9\linewidth]{{demo_P-0.4_0600_12.000}.eps}
		\caption{$t = \SI{12}{\second}$}
		\label{fig:demo_P-0.4_d}
	\end{subfigure}

	\vspace{\baselineskip}

	\begin{subfigure}{.48\textwidth}
		\centering
		\includegraphics[width=.9\linewidth]{{demo_P-0.4_0800_16.000}.eps}
		\caption{$t = \SI{16}{\second}$}
		\label{fig:demo_P-0.4_e}
	\end{subfigure}
	\begin{subfigure}{.5\textwidth}
		\centering
		\includegraphics[width=.9\linewidth]{{demo_P-0.4_1000_20.000}.eps}
		\caption{$t = \SI{20}{\second}$}
		\label{fig:demo_P-0.4_f}
	\end{subfigure}

	\caption{Simulation of the evacuation procedure of $N=600$ agents, through two $\SI{4.7}{\meter}$ wide exits, with a panic value of $p=0.4$, at six different time instants. The whole evacuation procedure takes $\SI{21.44}{\second}$. (a): The agents choose an exit and start moving. (b): The agents have divided into two groups. (c): The first agents have reached the exits. (d): Arch-like blocking of the exists has occurred. (e): Slow evacuation due to congestion around exits. (f): Almost all the agents have been evacuated.}
	\label{fig:demo_P-0.4}
\end{figure}

Next, in order to see the effect of different values of the panic level $p$ on the total evacuation time, multiple simulations were conducted with values of $p$ in the range $[0,0.9]$. For each value of $p$ in this range, $10$ different simulations were conducted and the mean evacuation of time was taken, in order to reduce the randomness induced by the random parameters and initializations of each simulation. The resulting evacuation times are presented in Figure \ref{fig:demo_P}. It is evident that an upward trend between the panic level and the evacuation time is present. The higher the panic level of the agents, the longer it takes them to evacuate the arena. This results also is consistent with empirical scenarios, where a crowd with a higher panic level takes longer to evacuate an area compared to a peaceful and relaxed evacuation. Additionally, the relation between the panic level and the evacuation time over the intervals $p \in [0,0.3]$ and $p \in [0.4,0.9]$ could clearly be approximated with a linear model.

\begin{figure}[ht!]
	\centering
	\includegraphics[width=.6\linewidth]{{demo_P}.eps}
	\caption{Relation between the panic level $p$ and the total evacuation time, increasing levels of panic leads to an increasing evacuation time, as expected. The evacuation time was calculated as the mean of $10$ different simulations, for each panic level.}
	\label{fig:demo_P}
\end{figure}


\subsection{Excitement Factor}
In order to determine how the excitement factor of the agent's affect their behaviour, simulations with different excitement factors $E_i$ were conduct. The excitement factor that was was the same for all agents in each simulation, and therefore will be henceforth denoted as just $E$. The simulations were conducted in an arena with two exits of width $\SI{4.7}{\meter}$ and $N=800$ agents.

The excitement factor $E \in (0,1)$, as it appears in \eqref{eq:Uik}, affects the importance each agents gives to the his distance from each exit in its evaluation. Furthermore, it affects the field of view of each agent for evaluating the available exits. A very low value of $E$ (close to $0$) would mean that each agent pays more attention to his distance from each available exit and that he has a very large field of view (close to $\ang{360}$)to look for an exit, whereas a very large value of $E$ (close to $1$) would mean that each agent pays less attention to his distance from each available exit and that he has a very narrow field o f view (close to $\ang{180}$) to look for an exit. Namely, a low value of $E$ means that each agent makes a well-informed choice regarding the best exit he can evacuate from, whereas a high value means that each makes a hastier and relatively ill-informed choice regarding the best exit he can evacuate from. In general, in an urgent evacuation scenario the excitement factor will be high, whereas in a relaxed non-urgent evacuation scenario the excitement factor will be low.

In order to see the effect of different values of the excitement factor $E$ on the evacuation procedure, two simulations were conducted. The first simulation had a low excitement factor of $E=0.2$, whereas the second simulation had a high excitement factor of $E=0.8$. All the other parameters of the two simulations including the random initialization of both were identical, in order to isolate the effect of the excitement factor on the evacuation procedure.

Four frames from each of the $\SI{60}{\second}$ simulations are presented in Figure \ref{fig:demo_E}. At $t=\SI{6}{\second}$, in Figure \ref{fig:demo_E-0.2_a} for the simulation with the low excitement factor, a group of agents separates from the main group (that is heading towards the green exit) and starts heading towards the teal exit. In Figure \ref{fig:demo_E-0.8_b}, for the simulation with the high excitement factor, very few agents have separated from the main group. At $t=\SI{16}{\second}$, in Figure \ref{fig:demo_E-0.2_c} for the simulation with the low excitement factor, a large arch-like blocking has occurred around the green exit while the smaller group of agents is heading towards the teal exit. In Figure \ref{fig:demo_E-0.8_d} for the simulation with the high excitement factor, the arch-like blocking around the green exit is even larger, since considerably fewer agents are heading towards the teal exit. At $t=\SI{26}{\second}$, in Figure \ref{fig:demo_E-0.2_e} for the simulation with the low excitement factor, the evacuation from the green exit is progressing slowly due to the high congestion and the second group of agents has just reach the teal exit. In Figure \ref{fig:demo_E-0.8_f} there is a similar situation around the green exit while the smaller second group has also just reached the teal exit. At $t=\SI{32}{\second}$, in Figure \ref{fig:demo_E-0.2_g} for the simulation with the low excitement factor, the evacuation is progressing with agents leaving the arena from both exits concurrently. In Figure \ref{fig:demo_E-0.8_h} for the simulation with the high excitement factor, the evacuation is progressing with agents leaving the arena solely through the green exit and the teal exit is not being used. The whole evacuation procedure, until all $800$ agents have evacuated the arena, takes $\SI{43.48}{\second}$ and $\SI{47.36}{\second}$ for the simulation with the low and high excitement factor respectively.

\begin{figure}[ht!]
	\begin{subfigure}{.48\textwidth}
		\centering
		\includegraphics[width=.9\linewidth]{{demo_E-0.2_0600_6.000}.eps}
		\caption{$E = 0.2, t = \SI{6}{\second}$}
		\label{fig:demo_E-0.2_a}
	\end{subfigure}
	\begin{subfigure}{.5\textwidth}
		\centering
		\includegraphics[width=.9\linewidth]{{demo_E-0.8_0600_6.000}.eps}
		\caption{$E = 0.8, t = \SI{6}{\second}$}
		\label{fig:demo_E-0.8_b}
	\end{subfigure}
	
	\vspace{\baselineskip}
	
	\begin{subfigure}{.48\textwidth}
		\centering
		\includegraphics[width=.9\linewidth]{{demo_E-0.2_1800_18.000}.eps}
		\caption{$E = 0.2, t = \SI{16}{\second}$}
		\label{fig:demo_E-0.2_c}
	\end{subfigure}
	\begin{subfigure}{.5\textwidth}
		\centering
		\includegraphics[width=.9\linewidth]{{demo_E-0.8_1800_18.000}.eps}
		\caption{$E = 0.8, t = \SI{16}{\second}$}
		\label{fig:demo_E-0.8_d}
	\end{subfigure}
	
	\vspace{\baselineskip}
	
	\begin{subfigure}{.48\textwidth}
		\centering
		\includegraphics[width=.9\linewidth]{{demo_E-0.2_2600_26.000}.eps}
		\caption{$E = 0.2, t = \SI{26}{\second}$}
		\label{fig:demo_E-0.2_e}
	\end{subfigure}
	\begin{subfigure}{.5\textwidth}
		\centering
		\includegraphics[width=.9\linewidth]{{demo_E-0.8_2600_26.000}.eps}
		\caption{$E = 0.8, t = \SI{26}{\second}$}
		\label{fig:demo_E-0.8_f}
	\end{subfigure}

	\vspace{\baselineskip}

	\begin{subfigure}{.48\textwidth}
		\centering
		\includegraphics[width=.9\linewidth]{{demo_E-0.2_3200_32.000}.eps}
		\caption{$E = 0.2, t = \SI{32}{\second}$}
		\label{fig:demo_E-0.2_g}
	\end{subfigure}
	\begin{subfigure}{.5\textwidth}
		\centering
		\includegraphics[width=.9\linewidth]{{demo_E-0.8_3200_32.000}.eps}
		\caption{$E = 0.8, t = \SI{32}{\second}$}
		\label{fig:demo_E-0.8_h}
	\end{subfigure}
	
	\caption{Simulation of the evacuation procedure of $N=800$ agents, through two $\SI{4.7}{\meter}$ wide exits, with excitement factors $E=0.2$ and $E=0.8$, at four different time instants. The whole evacuation procedure takes $\SI{43.48}{\second}$ and $\SI{47.36}{\second}$, respectively.}
	\label{fig:demo_E}
\end{figure}

In the simulation with the lower excitement factor the agents made more informed decisions regarding the utility of the exits available to them. As a result a portion of the agents chose to head for the teal exit the moment they identified the significant congestion that was building up in the green exit, even though the teal exit was significantly further away compared to the green exit. In contrast, in the simulation with the higher excitement factor most of the agents disregarded the second agent and opted to congest around the green exit which was closer to them. The fact that some of the agents headed for a new exit when its utility became sufficiently high is consistent with empirical observations of evacuation procedures of groups of people, as noted by~\citet{Helbing1997} and~\citet{Seneviratne1985}.
