\section{Introduction and Motivations}
\subsection{Introduction}
Evacuation of a building as a topic that has been studied many times,
e.g.\ by~\citet{Helbing2000}.
Most models don't include more than one exit, or have the agents move to a
fixed exit based on position (i.e.\ the closest exit). This might not be an
accurate model, because the choice of exit depends on other factors as well (Is
the exit well visible? Is the exit blocked by other people?).

We extend and concrete a model that accounts for these influencing factors,
namely the model proposed in~\citet{Wang2016}.

\subsection{Motivation}
In general, having more emergency exits should help with evacuation times due
to reduced congestion. However, this is only the case if the choice of exit is
distributed evenly between exits. If the people don't choose among the exits
evenly enough, some exits might be overused and get congested.

To optimize the evacuation time of a building or similar structure, it is
essential that exits are placed such that they are used similarly often, and
for this a model that accounts for the choice of exit is essential.

Our work might help with designing floor plans for buildings, by allowing the
architect to experiment with positions of emergency exits to determine the
optimal placement.
